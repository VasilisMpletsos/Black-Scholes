\chapter*{\center \Large  Περίληψη}

%%%
~\\[1cm]%REMOVE THIS
\noindent\textbf{Περίληψη:} % Provide your abstract

Η παρούσα εργασία διερευνά τη χρήση FPGA (Field Programmable Gate Arrays) για την πρόβλεψη ομολόγων
με στόχο την επίτευξη πιο αποδοτικής και ταχύτερης επεξεργασίας δεδομένων. Η προσέγγιση αυτή βασίζεται 
στην ικανότητα των FPGA να εκτελούν παράλληλους υπολογισμός. 'Ετσι επιτυγχάνεται επιτάχυνση αλγορίθμων πρόβλεψης
με σκοπό την μειώση του χρόνο εκτέλεσης και την κατανάλωση ενέργειας σε σύγκριση με παραδοσιακές αρχιτεκτονικές
με χρήση CPU, multi-threading η multi-processing που είναι σημαντικά πιο αργές ή η χρήση GPU για την ταχύτερη παραλληλοποίηση αλλά με μεγάλος κόστος ενέργειας.
Τα αποτελέσματα δείχνουν ότι η χρήση FPGA μπορεί να προσφέρει σημαντικά πλεονεκτήματα σε εφαρμογές χρηματοοικονομικής ανάλυσης,
κάνωντας την τεχνολογία αυτή απαραίτητη για την χρήση της ωστε να αποκτηθεί το πάνω χέρι έναντι των απαιτήσεων της σύγχρονης αγοράς και των ανταγωνιστών.


%%%%%%%%%%%%%%%%%%%%%%%%%%%%%%%%%%%%%%%%%%%%%%%%%%%%%%%%%%%%%%%%%%%%%%%%%s
~\\[1cm]
\noindent % Provide your key words
\textbf{Λέξεις κλειδιά:} FPGA, Accelerated Computing, Χρηματηστήριο, Επιτάχυνση Υλικού, GPU, CPU 