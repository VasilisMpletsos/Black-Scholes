\chapter{Αλγόριθμος Black-Scholes}
\label{chap:black_scholes}

\section{Εισαγωγή}
\begin{frame}
  \frametitle{Εισαγωγή}
  \begin{itemize}
    \item Ο αλγόριθμος Black-Scholes είναι ένα μοντέλο για την τιμολόγηση ευρωπαϊκών δικαιωμάτων προαίρεσης.
    \item Ο αλγόριθμος αυτός αναπτύχθηκε το 1973 από τους Fischer Black, Myron Scholes και Robert Merton.
    \item Η εξίσωση Black-Scholes είναι μια διαφορική εξίσωση που περιγράφει την εξέλιξη της τιμής ενός δικαιώματος προαίρεσης με την πάροδο του χρόνου.
  \end{itemize}

  \begin{block}{Σημαντικότητα}
    \begin{itemize}
      \item Ο αλγόριθμος Black-Scholes έχει επηρεάσει σημαντικά την ανάπτυξη των χρηματοοικονομικών αγορών.
      \item Έχει κερδίσει το βραβείο Νόμπελ Οικονομίας το 1997 για την εφαρμογή του στη χρηματοοικονομική θεωρία.
    \end{itemize}
  \end{block}
\end{frame}

\section{Εξίσωση Black-Scholes}
\begin{frame}
  \frametitle{Εξίσωση Black-Scholes}
  \begin{itemize}
    \item Η εξίσωση Black-Scholes δίνεται από την εξίσωση:
      \[
        \frac{\partial V}{\partial t} + \frac{1}{2} \sigma^2 S^2 \frac{\partial^2 V}{\partial S^2} + r S \frac{\partial V}{\partial S} - r V = 0
      \]
      όπου:
      \begin{itemize}
        \item \(V\) είναι η τιμή του δικαιώματος προαίρεσης,
        \item \(S\) είναι η τιμή του υποκείμενου περιουσιακού στοιχείου,
        \item \(t\) είναι ο χρόνος,
        \item \(r\) είναι το επιτόκιο,
        \item \(\sigma\) είναι η μεταβλητότητα.
      \end{itemize}
  \end{itemize}
  \begin{block}{Σημαντικές παράμετροι}
    \begin{itemize}
      \item \(S\): Τιμή υποκείμενου περιουσιακού στοιχείου
      \item \(K\): Τιμή εξάσκησης
      \item \(T\): Χρόνος λήξης
      \item \(r\): Ετήσιο επιτόκιο
      \item \(\sigma\): Μεταβλητότητα
    \end{itemize}
  \end{block}
\end{frame}

\section{Αλγόριθμος Black-Scholes}
\begin{frame}
  \frametitle{Αλγόριθμος Black-Scholes}
  \begin{itemize}
    \item Ο αλγόριθμος Black-Scholes υπολογίζει την τιμή ενός ευρωπαϊκού δικαιώματος προαίρεσης με βάση τις παραμέτρους του υποκείμενου περιουσιακού στοιχείου.
    \item Η τιμή του δικαιώματος προαίρεσης δίνεται από την εξίσωση:
      \[
        C = S N(d_1) - K e^{-rT} N(d_2)
      \]
      όπου:
      \[
        d_1 = \frac{\ln(S/K) + (r + \sigma^2/2)T}{\sigma\sqrt{T}}
      \]
      \[
        d_2 = d_1 - \sigma\sqrt{T}
      \]
  \end{itemize}
  \begin{block}{Σημαντικές συναρτήσεις}
    \begin{itemize}
      \item \(N(d)\): Συνάρτηση κατανομής κανονικής κατανομής
    \end{itemize}
  \end{block}
  \begin{block}{Σημαντικές παράμετροι}
    \begin{itemize}
      \item \(C\): Τιμή δικαιώματος προαίρεσης
      \item \(K\): Τιμή εξάσκησης
      \item \(T\): Χρόνος λήξης
      \item \(r\): Ετήσιο επιτόκιο
      \item \(\sigma\): Μεταβλητότητα
    \end{itemize}
  \end{block}
\end{frame}

\section{Εφαρμογές}
\begin{frame}
  \frametitle{Εφαρμογές}
  \begin{itemize}
    \item Ο αλγόριθμος Black-Scholes χρησιμοποιείται ευρέως για την τιμολόγηση δικαιωμάτων προαίρεσης σε χρηματοοικονομικές αγορές.
    \item Χρησιμοποιείται επίσης για τη διαχείριση κινδύνου και την εκτίμηση της μεταβλητότητας των υποκείμενων περιουσιακών στοιχείων.
    \item Ο αλγόριθμος έχει επηρεάσει τη χρηματοοικονομική θεωρία και πρακτική, οδηγώντας σε νέες στρατηγικές και προϊόντα.
  \end{itemize}
  \begin{block}{Σημαντικές εφαρμογές}
    \begin{itemize}
      \item Τιμολόγηση δικαιωμάτων προαίρεσης
      \item Διαχείριση κινδύνου
      \item Εκτίμηση μεταβλητότητας
    \end{itemize}
  \end{block}
  \begin{block}{Σημαντικά προϊόντα}
    \begin{itemize}
      \item Ευρωπαϊκά δικαιώματα προαίρεσης
      \item Αμερικανικά δικαιώματα προαίρεσης
      \item Δικαιώματα προαίρεσης σε μετοχές και δείκτες
    \end{itemize}
  \end{block}
\end{frame}
\section{Συμπεράσματα}
\begin{frame}
  \frametitle{Συμπεράσματα}
  \begin{itemize}
    \item Ο αλγόριθμος Black-Scholes είναι ένα σημαντικό εργαλείο για την τιμολόγηση δικαιωμάτων προαίρεσης.
    \item Έχει επηρεάσει τη χρηματοοικονομική θεωρία και πρακτική, οδηγώντας σε νέες στρατηγικές και προϊόντα.
    \item Η κατανόηση του αλγορίθμου είναι κρίσιμη για τη διαχείριση κινδύνου και την εκτίμηση της μεταβλητότητας των υποκείμενων περιουσιακών στοιχείων.
  \end{itemize}
  \begin{block}{Σημαντικά σημεία}
    \begin{itemize}
      \item Ο αλγόριθμος Black-Scholes είναι ένα σημαντικό εργαλείο για την τιμολόγηση δικαιωμάτων προαίρεσης.
      \item Έχει επηρεάσει τη χρηματοοικονομική θεωρία και πρακτική.
      \item Η κατανόηση του αλγορίθμου είναι κρίσιμη για τη διαχείριση κινδύνου.
    \end{itemize}
  \end{block}
  \begin{block}{Σημαντικές παρατηρήσεις}
    \begin{itemize}
      \item Ο αλγόριθμος Black-Scholes έχει περιορισμούς και υποθέσεις που πρέπει να ληφθούν υπόψη.
      \item Η εφαρμογή του αλγορίθμου απαιτεί προσεκτική ανάλυση των παραμέτρων και των συνθηκών της αγοράς.
    \end{itemize}
  \end{block}
\end{frame}
