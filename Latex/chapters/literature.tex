\chapter{Επισκόπηση της βιβλιογραφίας}
\label{ch:lit_rev}


\section{Εισαγωγή}

Η τιμολόγηση χρηματοοικονομικών παραγώγων με το μοντέλο Black–Scholes έχει αποτελέσει αντικείμενο ενδιαφέρουσας έρευνας τις τελευταίες δεκαετίες,
τόσο σε θεωρητικό όσο και σε πρακτικό επίπεδο.
Η ανάγκη για ταχύτερους και αποδοτικότερους υπολογισμούς οδήγησε στην ανάπτυξη πληθώρας υλοποιήσεων, αξιοποιώντας διάφορες υπολογιστικές πλατφόρμες
όπως CPUs, GPUs και FPGAs. Στη διεθνή βιβλιογραφία συναντώνται διαφορετικές προσεγγίσεις, που περιλαμβάνουν αριθμητικές μεθόδους (όπως πεπερασμένες διαφορές και Monte Carlo),
καθώς και βελτιστοποιήσεις σε επίπεδο υλικού και λογισμικού.
Στη συνέχεια παρουσιάζονται διάφορες ενδεικτικές εργασίες και δημοσιεύσεις που έχουν συμβάλει σημαντικά στην εξέλιξη της βελτιστοποίησης εκτέλεσης των υλοποιήσεων,
αναδεικνύοντας τις τεχνολογικές τάσεις και τις καινοτομίες που έχουν προταθεί διαχρονικά.

\section{On Comparing Financial Option Price Solvers on FPGA}

Η εργασία των \cite{comparing_option_price_solvers_fpga} παρουσιάζει για πρώτη φορά ένα καινοτόμο πλαίσιο για τη συγκριτική αξιολόγηση διαφορετικών αριθμητικών
μεθόδων τιμολόγησης χρηματοοικονομικών δικαιωμάτων προαίρεσης (options) που υλοποιούνται σε συσκευές FPGA.
Αντί να επικεντρώνεται αποκλειστικά στην επιτάχυνση (\textit{speed-up}) κάθε μεθόδου έναντι της αντίστοιχης υλοποίησής της σε λογισμικό, η μελέτη εξετάζει
τη σχέση μεταξύ \textbf{ταχύτητας} και \textbf{ακρίβειας}, αναζητώντας τη βέλτιστη μέθοδο για εκτέλεση σε FPGA.

\paragraph{Μεθοδολογία.}  
Η εργασία παρουσιάζεις ένα ενιαίο πλαίσιο αξιολόγησης (\textit{framework}) που περιλαμβάνει:
\begin{itemize}
  \item \textbf{Χρόνους εκτέλεσης (Execution Time):} συνάρτηση της συχνότητας λειτουργίας, του αριθμού επαναλήψεων και του πλήθους των πυρήνων.
  \item \textbf{Ακρίβεια (Accuracy):} μετρούμενη μέσω του \textit{Root Mean Squared Error (RMSE)} σε σχέση με τιμή αναφοράς (αναλυτική ή αριθμητική υψηλής ακρίβειας).
  \item \textbf{Χρήση πόρων (Resource Utilisation):} εκτίμηση της κατανάλωσης LUTs και DSPs
\end{itemize}
Η μεθοδολογία εφαρμόζεται σε πέντε υλοποιήσεις για ευρωπαϊκά και αμερικανικά δικαιώματα προαίρεσης: 
\begin{itemize}
    \item \textbf{Binomial Tree}
    \item \textbf{Trinomial Tree}
    \item \textbf{Explicit Finite Difference}
    \item \textbf{Quadrature}
    \item \textbf{Monte Carlo}
\end{itemize}

Οι υλοποιήσεις πραγματοποιήθηκαν σε FPGA \textit{Xilinx Virtex-4} (μοντέλα SX55, LX160 και FX100) με απλή ακρίβεια κινητής υποδιαστολής. Τα κύρια ευρήματα που συνοψίζουν οι συγγραφείς έιναι τα εξής:
\begin{itemize}
  \item Η μέθοδος \textbf{Quadrature} παρουσίασε την ταχύτερη σύγκλιση και τη μεγαλύτερη ακρίβεια για τα ευρωπαϊκά δικαιώματα, επιτυγχάνοντας το χαμηλότερο RMSE σε ελάχιστο χρόνο.
  \item Για αμερικανικά δικαιώματα, η \textbf{Quadrature} παρέμεινε η ταχύτερη, αλλά η \textbf{Finite Difference} απέδωσε ακριβέστερα αποτελέσματα σε περιπτώσεις που απαιτείται υψηλή ακρίβεια.
  \item Οι μέθοδοι \textbf{Binomial} και \textbf{Trinomial Tree} είχαν παρόμοια συμπεριφορά, με μικρότερη ακρίβεια αλλά σταθερή απόδοση.
  \item Η \textbf{Monte Carlo} μέθοδος, παρότι εμφάνισε σημαντικές επιταχύνσεις σε προηγούμενες μελέτες, αποδείχθηκε έως και \textbf{100 φορές λιγότερο ακριβής} (σε λογαριθμική κλίμακα RMSE),
  καθιστώντας την ακατάλληλη για ακριβή αποτελέσματα, εκτός αν δεν υπάρχει άλλη διαθέσιμη μέθοδος.
\end{itemize}

\begin{figure}[h!]
  \centering
  \includegraphics[width=1.0\textwidth]{figures/chapter4/options_pricing_time_comparison.png}
  \caption{Σύγκριση χρόνων εκτέλεσης και ακρίβειας για διάφορες αριθμητικές μεθόδους τιμολόγησης options σε FPGA.}
  \label{fig:options_pricing_time_comparison}
\end{figure}

Η εργασία αυτή συμβάλλει καθοριστικά στη βιβλιογραφία με τους εξής τρόπους:
\begin{itemize}
  \item Εισάγει ένα \textbf{γενικό πλαίσιο αξιολόγησης} που συνδυάζει μετρήσεις ταχύτητας, ακρίβειας και κατανάλωσης πόρων για FPGA εφαρμογές.
  \item Παρέχει μια \textbf{συγκριτική κατάταξη αριθμητικών μεθόδων} για το πρόβλημα Black–Scholes, βοηθώντας στον σχεδιασμό βελτιστοποιημένων FPGA αρχιτεκτονικών.
  \item Τεκμηριώνει ότι η \textbf{Monte Carlo} προσέγγιση, παρά την υψηλή παραλληλία, υπολείπεται σε ακρίβεια και σύγκλιση, ειδικά σε εφαρμογές υψηλών απαιτήσεων.
\end{itemize}

\section{An FPGA-based Parallel Processor for Black-Scholes Option Pricing Using Finite Differences Schemes}

Η εργασία \cite{fpga_parallel_processor_black_scholes_papaeustathiou} παρουσιάζει μία καινοτόμο προσέγγιση
για την επιτάχυνση των υπολογισμών τιμολόγησης χρηματοοικονομικών παραγώγων, βασισμένη σε υλοποίηση με χρήση FPGA.
Η μελέτη εστιάζει στην επίλυση της μερικής διαφορικής εξίσωσης του μοντέλου Black-Scholes, μέσω δύο αριθμητικών μεθόδων πεπερασμένων διαφορών:
\begin{itemize}
    \item Της ρητής (explicit)
    \item Της Crank–Nicholson (ημι-ρητής) μεθόδου
\end{itemize}

Η συγγραφείς προτείνουν μια αρχιτεκτονική η οποία αποτελείται από αρκετούς παράλληλους πυρήνες (\textit{cores}) συνδεδεμένους
μέσω δακτυλιοειδούς διαύλου \textbf{ring bus}, με κάθε πυρήνα να διαθέτει μια μονάδα κινητής υποδιαστολής (FPU), τοπική μνήμη
και ελεγκτή μνήμης. 
Οι δύο μέθοδοι πεπερασμένων διαφορών έχουν υλοποιηθεί πλήρως εντός της FPGA:
\begin{itemize}
  \item Η \textbf{ρητή μέθοδος} χρησιμοποιεί προωθημένες διαφορές ως προς τον χρόνο και κεντρικές διαφορές ως προς την τιμή του υποκείμενου τίτλου, με έμφαση στην υπολογιστική αποδοτικότητα.
  \item Η \textbf{Crank–Nicholson μέθοδος} παρέχει μεγαλύτερη σταθερότητα και ταχύτερη σύγκλιση, αλλά απαιτεί επίλυση τριδιαγώνιων συστημάτων εξισώσεων, η οποία επιταχύνεται
  μέσω μιας παραλλαγής του αλγορίθμου με textbf{Cyclic Odd-Even Reduction}.
\end{itemize}

Η αξιολόγηση της προτεινόμενης αρχιτεκτονικής εκτελέστηκε σε FPGA \textbf{Xilinx Virtex-5}, με σύγκριση έναντι ενός επεξεργαστή \textbf{Intel Core 2 Duo 2GHz}.
Τα αποτελέσματα έδειξαν σχεδόν γραμμική αύξηση της απόδοσης με τον αριθμό των πυρήνων. Συγκεκριμένα:
\begin{itemize}
  \item Για τη ρητή μέθοδο, επιτεύχθηκε επιτάχυνση έως και \textbf{8x}.
  \item Για τη μέθοδο Crank–Nicholson, επιτεύχθηκε επιτάχυνση περίπου \textbf{5x}.
\end{itemize}
Οι συγγραφείς επισημαίνουν ότι τα αποτελέσματα αυτά επιτυγχάνονται παρά το γεγονός ότι η FPGA λειτουργεί σε 14 φορές χαμηλότερη συχνότητα από τον επεξεργαστή, γεγονός που καταδεικνύει τη σημαντική αποτελεσματικότητα της παράλληλης αρχιτεκτονικής.

Η εργασία συνεισφέρει σημαντικά στη διεθνή βιβλιογραφία, καθώς:
\begin{itemize}
  \item Παρουσιάζει την \textbf{πρώτη επεκτάσιμη FPGA αρχιτεκτονική} που υλοποιεί και τις δύο μεθόδους (ρητή και Crank–Nicholson) για την επίλυση του μοντέλου Black–Scholes.
  \item Εισάγει μια βελτιστοποιημένη παραλλαγή του αλγορίθμου \textbf{Odd-Even Reduction} για την επίλυση τριδιαγώνιων συστημάτων.
  \item Δείχνει ότι η επιτάχυνση κλιμακώνεται σχεδόν γραμμικά με τους διαθέσιμους πόρους της FPGA.
\end{itemize}

\section{FPGA-Based Design of Black Scholes Financial Model for High Performance Trading}

Η εργασία \cite{fpga_black_scholes_choo} παρουσιάζει μια υλοποίηση υλικού για το μοντέλο Black-Scholes, εστιασμένη στην αποδοτική τιμολόγηση Ευρωπαϊκών call options σε FPGA, με στόχο υψηλές επιδόσεις σε συστήματα συναλλαγών.
Η μελέτη εστιάζει στην υλοποίηση του μοντέλου Black-Scholes, το οποίο παρέχει ταχύτητα και ακρίβεια σε σχέση με άλλα μοντέλα όπως:
\begin{itemize}
    \item Το μοντέλο Monte Carlo (προσομοίωση με τυχαίες δειγματοληψίες)
    \item Το μοντέλο Binomial Tree (δενδρική προσέγγιση διακριτού χρόνου)
\end{itemize}

Οι συγγραφείς προτείνουν μια αρχιτεκτονική διαιρεμένη σε τρία κύρια μπλοκ:
\begin{itemize}
  \item Το \textbf{D1D2 block}, το οποίο υπολογίζει τις παραμέτρους $d_1$ και $d_2$ χρησιμοποιώντας πράξεις διπλής ακρίβειας (64-bit IEEE floating-point), με IPs για διαίρεση, τετραγωνική ρίζα, πολλαπλασιασμό, λογάριθμο και πρόσθεση/αφαίρεση.
  \item Το \textbf{CDF block} (Cumulative Distribution Function), το οποίο υπολογίζει τη σωρευτική κανονική κατανομή $N(d_1)$ και $N(d_2)$ με πολυωνυμική προσέγγιση ακρίβειας 6 δεκαδικών ψηφίων, χρησιμοποιώντας πολλαπλασιαστές, προσθέσεις, εκθετική και διαίρεση.
  \item Το \textbf{Option Price block}, το οποίο συνδυάζει τα αποτελέσματα για τον υπολογισμό της τελικής τιμής του call option.
\end{itemize}

Η αξιολόγηση της προτεινόμενης αρχιτεκτονικής εκτελέστηκε σε FPGA \textbf{Altera Stratix V} (5SGXEA7K2F40C2), με σύγκριση ως προς άλλα μοντέλα και FPGA οικογένειες (Stratix III, IV).
Τα αποτελέσματα έδειξαν:
\begin{itemize}
  \item Μέγιστη συχνότητα \textbf{179 MHz}.
  \item Εκτέλεση \textbf{180 εκατομμυρίων συναλλαγών ανά δευτερόλεπτο}, μετά από αρχική καθυστέρηση \textbf{208 κύκλων ρολογιού}.
  \item Χρήση πόρων: 68.148 logic elements, 27.705 memory bits και 261 DSP blocks.
  \item Αποτυχία υλοποίησης σε Stratix III λόγω έλλειψης πόρων, ενώ η Stratix IV πέτυχε 139 MHz.
\end{itemize}
Η επαλήθευση έγινε με μοντέλα σε MATLAB και C/C++, επιβεβαιώνοντας την ακρίβεια και λειτουργικότητα.

Η εργασία συνεισφέρει σημαντικά στη διεθνή βιβλιογραφία, καθώς:
\begin{itemize}
  \item Παρουσιάζει μια \textbf{αποδοτική υλοποίηση υλικού} του μοντέλου Black-Scholes σε FPGA, με έμφαση σε υψηλή ταχύτητα και χαμηλή κατανάλωση.
  \item Συγκρίνει το Black-Scholes με Monte Carlo και Binomial, τονίζοντας τα πλεονεκτήματα σε ταχύτητα και ακρίβεια για πραγματικού χρόνου εφαρμογές.
  \item Δείχνει ότι η FPGA υλοποίηση υπερτερεί των υπερυπολογιστών σε όρους χώρου, ισχύος και κόστους, με δυνατότητα επέκτασης σε πιο σύνθετα χρηματοοικονομικά μοντέλα.
\end{itemize}

\section{Black-Scholes Option Pricing on Intel CPUs and GPUs}

Η εργασία των \cite{black_scholes_option_pricing_intel_cpus_gpus} επικεντρώνεται στην υλοποίηση και βελτιστοποίηση του αλγορίθμου τιμολόγησης δικαιωμάτων Black–Scholes
σε ετερογενή περιβάλλοντα υπολογισμού, χρησιμοποιώντας τη γλώσσα προγραμματισμού SYCL (Data Parallel C++ – DPC++) στο πλαίσιο του οικοσυστήματος \textit{oneAPI} της Intel.
Παρουσιάζει έτσι λοιπόν την ενδιαφέρουσα σύγκριση μεταξύ CPU, GPU, FPGA απο το οποίο εμπνεύστηκαμε να κάνουμε και σε αυτή την εργασία για να συγκρίνουμε την απόδοση βελτιστοποίησης του Vitis. 

Η εργασία ξεκινά με μία κλασική υλοποίηση του μοντέλου Black–Scholes σε \textbf{C++ με χρήση OpenMP} και στη συνέχεια παρουσιάζει τη σταδιακή βελτιστοποίηση του 
κώδικα για πολυπύρηνους επεξεργαστές Intel Xeon. Οι κυριότερες τεχνικές βελτιστοποίησης που εφαρμόστηκαν περιλαμβάνουν:
\begin{itemize}
\item \textbf{Διανυσματοποίηση βρόχων (Loop Vectorization)} για εκμετάλλευση της \textbf{AVX-512} (Advanced Vector Extensions 512), η οποία είναι ένα σύνολο εντολών SIMD 512-bit για x86 επεξεργαστές.
Η χρήση AVX-512 επιτρέπει την ταυτόχρονη επεξεργασία μεγάλων τμημάτων δεδομένων σε έναν κύκλο ρολογιού, αυξάνοντας σημαντικά την απόδοση σε εφαρμογές υψηλών απαιτήσεων όπως HPC, data analytics και AI.
\item \textbf{Μείωση ακρίβειας (Precision Reduction)} με χρήση αριθμητικών τύπων \texttt{float} και περιορισμένων bits. Αυτή η τεχνική μειώνει σημαντικά την κατανάλωση μνήμης αλλά και το φόρτο μεταφοράς δεδομένων, οδηγώντας σε ταχύτερους υπολογισμούς.
Είναι και μία απο τις κλασσικές τεχνικές στην επιτάχυνση εφαρμογών AI και machine learning, "θυσιάζοντας" λίγη ακρίβεια για χάρη της ταχύτητας.
\item \textbf{Παράλληλη εκτέλεση (OpenMP Parallelism)} σε 48 φυσικούς πυρήνες.
\item \textbf{NUMA-Friendly κατανομή μνήμης}, ώστε να αποφευχθεί μη ομοιόμορφη πρόσβαση στη μνήμη.
\end{itemize}

Μετά την ολοκλήρωση των παραπάνω βελτιστοποιήσεων, οι συγγραφείς αναφέρουν οτι ο χρόνος εκτέλεσης μειώθηκε από 3.062 sec σε 0.022 sec, επιτυγχάνοντας συνολική επιτάχυνση περίπου 140x.  
Στη συνέχεια, ο κώδικας μεταφέρθηκε στη SYCL/DPC++ υλοποίηση, με σκοπό την εκτέλεση του ίδιου προγράμματος τόσο σε CPUs όσο και σε GPUs.
Οι συγγραφείς χρησιμοποίησαν τον μηχανισμό \textbf{Buffers \& Accessors} για τη διαχείριση μνήμης και στη συνέχεια δοκίμασαν εναλλακτικές στρατηγικές με \textbf{Unified Shared Memory (USM)},
τόσο σε implicit όσο και explicit μορφές, προκειμένου να μειωθεί το overhead μεταφοράς δεδομένων μεταξύ host και συσκευής.

Οι δοκιμές πραγματοποιήθηκαν σε υπερυπολογιστικό κόμβο με \textbf{Intel Xeon Platinum 8260L} αλλά και σε \textbf{GPU Intel Iris Xe MAX}.
Τα πειραματικά αποτελέσματα τους έδειξαν ότι:
\begin{itemize}
  \item Η DPC++ υλοποίηση σε CPU ήταν μόλις \textbf{10\% πιο αργή} από τον άκρως βελτιστοποιημένο OpenMP κώδικα (0.024 sec έναντι 0.022 sec).
  \item Στις GPUs, η απόδοση περιορίστηκε από το \textbf{εύρος ζώνης της μνήμης} (DRAM bandwidth), επιβεβαιώνοντας ότι η εφαρμογή είναι \textit{memory-bound}.
  \item Η χρήση \textbf{explicit USM} με κοινή μνήμη για όλα τα batches μείωσε το overhead αντιγραφής δεδομένων κατά τρεις φορές.
\end{itemize}
Σημαντικό είναι ότι το DPC++ πρόγραμμα εκτελείται απευθείας και στις δύο αρχιτεκτονικές χωρίς αλλαγές στον κώδικα, παρέχοντας ένα ενιαίο και φορητό προγραμματιστικό μοντέλο.
 
Η εργασία αναδεικνύει ότι είναι δυνατή η ανάπτυξη ενός ενιαίου και φορητού κώδικα υψηλής απόδοσης για ετερογενή περιβάλλοντα. Τα κύρια συμπεράσματα είναι:
\begin{itemize}
  \item Η \textbf{SYCL/DPC++ πλατφόρμα} μπορεί να προσεγγίσει την απόδοση native OpenMP εφαρμογών σε CPUs.
  \item Η \textbf{φορητότητα κώδικα} επιτρέπει την εκτέλεση σε GPUs με αποδεκτή απόδοση χωρίς ανασχεδιασμό του αλγορίθμου.
  \item Οι \textbf{τεχνικές βελτιστοποίησης} (vectorization, precision control, NUMA-awareness) παραμένουν κρίσιμες για επίτευξη κορυφαίας απόδοσης.
\end{itemize}

Αυτή την φορητότητα όμως πλέον υποστηρίζει και εγγενώς η Vitis, όπως θα δούμε και στην συνέχεια των δικών μας αναλύσεων και συγκρίσεων.

\section{Σημασία της Παρούσας Εργασίας}

Παρά τον πλούτο της υπάρχουσας βιβλιογραφίας, η παρούσα εργασία έρχεται να "ενημερώσει" πάνω σε αυτό το θέμα με σύγχρονα και ανανεωμένα εργαλεία 
και να προσφέρει νέες προοπτικές στη μελέτη της επιτάχυνσης του μοντέλου Black–Scholes με χρήση FPGAs και σύγχρονων υπολογιστικών πλατφορμών.
Συγκεκριμένα:

\begin{itemize}
    \item \textbf{Επικαιροποίηση της βιβλιογραφίας:} Οι περισσότερες σχετικές δημοσιεύσεις βασίζονται σε υλοποιήσεις και συγκρίσεις με παλαιότερες γενιές υλικού (παλαιά CPUs, GPUs και FPGAs).
    Η ταχεία εξέλιξη των σύγχρονων GPU (όπως οι γενιές NVIDIA RTX 30**, 40** και η νεότερη 50**) έχει αλλάξει δραστικά το τοπίο των επιταχυντών σε κάρτες γραφικών, καθιστώντας αναγκαία τη σύγκριση με τις τρέχουσες τεχνολογίες.
    \item \textbf{Σύγκριση C++ και Python:} Η βιβλιογραφία εστιάζει κυρίως σε υλοποιήσεις C/C++, παραβλέποντας τη ραγδαία εξάπλωση της Python ως κυρίαρχης γλώσσας στην επιστημονική και χρηματοοικονομική ανάλυση.
    Η εργασία μας εξετάζει και συγκρίνει την απόδοση μεταξύ C/C++ και Python, προσφέροντας πρακτικές απαντήσεις για την επιλογή γλώσσας σε πραγματικές εφαρμογές.
    \item \textbf{Vitis HLS και φορητότητα:} Ενώ οι περισσότερες προηγούμενες εργασίες υλοποιούν αλγορίθμους σε χαμηλού επιπέδου HDL, η δική μας προσέγγιση αξιοποιεί το \textbf{Vitis HLS} της Xilinx,
    επιτρέποντας την ανάπτυξη και βελτιστοποίηση σε C/C++ με αυτόματη μεταγλώττιση σε υλικό. Αυτό προσφέρει σημαντική \textbf{φορητότητα}, ταχύτερη ανάπτυξη και ευκολότερη συντήρηση του κώδικα.
    \item \textbf{Πειραματική σύγκριση με σύγχρονες αρχιτεκτονικές:} Η εργασία μας περιλαμβάνει μετρήσεις και συγκρίσεις απόδοσης σε σύγχρονες GPUs (όπως NVIDIA 3060), CPUs και FPGAs,
    προσφέροντας μια ρεαλιστική εικόνα των δυνατοτήτων κάθε πλατφόρμας με βάση τα σημερινά δεδομένα.
\end{itemize}

Συνολικά, η εργασία μας βασίζεται σε όλες τις προηγούμενες αναφερθείσες δημοσιεύσεις και επικαιροποιεί την σύγκριση, εισάγοντας σύγχρονες τεχνολογίες, πρακτικές συγκρίσεις και καινοτόμες μεθοδολογίες
που ανταποκρίνονται στις απαιτήσεις της σημερινής βιομηχανίας και έρευνας.
